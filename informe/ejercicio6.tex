Como ya menciona anteriormente, en la sección de la intruducción comentamos nuestras decisiones en cuanto a las estructuras de datos que utiliza la arquitectura de intel para el manejo de memoria y tareas.\\

En este ejercicio, lo primero que hicimos fue trabajar con el \textit{array} global de \textit{TSS} -llamado \textit{tsss}- que se encuentra en el archivo \textit{tss.c}. El mismo hace uso del \textit{struct tss} que se encuentra definido en \textit{tss.h}, que fue provisto por la cátedra. Este \textit{array} contendra las \textit{TSS} de todas las tareas, incluyendo a las tareas \textit{idle} e \textit{inicial}. Básicamente, siguiendo las decisiones que describimos durante la introducción, completamos los datos de las 7 \textit{TSS} con los datos correspondientes a cada tarea.\\

A su vez, volvimos al archivo \textit{gdt.c} donde se encuentra otro \textit{array} global, el \textit{array} de la GDT. Aquí, lo que hicios fue tomar las entradas 1-7 para utilizarlas como descriptores de \textit{TSS}. En estos descriptores, pudimos volcar toda la información correspondiente a cada tarea, excepto por la dirección base de las \textit{TSS}, ya que estas están definidas en el archivo \textit{tss.c}, y hasta que no se compile, no se puede saber sus direcciones de memoria.\\

Para poder completar los descriptores de \textit{TSS} en la GDT con las direcciones bases de las \textit{TSS}, solo nos queda resolver el problema en tiempo de ejecución (por lo ya explicado en el párrafo anterior). Para ello, invocamos desde el \textit{kernel.asm} a la función \textit{inicializar\_tsss} que está definida en el \textit{sched.c}. Esta función recorre todos los elementos del \textit{array tsss}, y utiliza la dirección de memoria de los mismos para setear los campos faltantes en el descriptor correspondiente en la GDT.\\

Ya aquí, sólo nos falta saltar a la tarea \textit{IDLE}. Como sabemos cual es el índice del descriptor de tarea que queremos utilizar (pues, no solo hemos comentado cual era en la introducción de este informe, sino que además lo acabamos de configurar), con utilizar la instrucción "jmp INDICE\_TAREA\_IDLE:saraza" estaremos comenzando todo el procesos de la arquitectura de intercambio de tarea. Previamente, hay que cargar en el registro \textit{tr} el índice del descriptor de tarea de la tarea inicial, para que al comenzar el proceso de intercambio de tarea, el contexto actual tenga una \textit{TSS} valida donde poder guardar la información, a pesar de que no volveremos a conmutar hacia ella nunca más.\\

También vale la pena aclarar, que estos "índices" de descriptores de tareas han de tener la \textit{RTL} correspondiente al nivel de privilegios con el que se correrá la tarea. Para más detalles, se puede ver la sección de introducción de este documento.