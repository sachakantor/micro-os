Aprovechando los \textit{structs} \textit{gdt\_entry} y \textit{gdt\_descriptor} provistos por la cátedra en el archivo \textit{gdt.h}, la tarea de completar la GDT con 5 segmentos de datos no conllevó una gran dificultad. Dichos \textit{structs} nos facilitan la configuración de los distintos bits de las estructuras de manejo de memoria utilizadas por la arquitectura de intel, ya que podemos trabajar con los nombres de los campos de estas estructuras olvidandonos de que bits especificos se deben setear.

Siguiendo con nuestras decisiones comentadas durante la introducción, en el archivo \textit{gdt.c} nos encontramos con un \textit{array} global (declarado fuera de cualquier funcion, es decir, sus datos serán escritos en el binario del \text{kernel}) de descriptores de segmentos llamado \textit{gdt}. En el mismo, inicializamos los primeros 10 elementos con descriptores nulos, y luego incializamos los 5 segmentos pedidos por el enunciado del ejercicio (codigo, datos -para supervisor y usuario- y video -solo para supervisor). Para más detalles sobre los datos de los descriptores, y sus índices se enunciados en la introducción.\\

Luego nos encontramos con otra variable global, \textit{GDT\_DESC}, que es del tipo \textit{gdt\_descriptor}. El mismo se inicializa con el tamaño del \text{array gdt} y la dirección de memoria del mismo (básicamente, la información que debe tener el registro \textit{gdtr} para pasar a modo protegido).\\

Luego, ya en el \textit{kernel.asm}, debemos empezar el proceso de pasar a modo protegido. Al ya tener en el binario del \textit{kernel} la estructura de la GDT (en el \textit{array} que configuramos arriba), y también tener el selector de la GDT en \textit{GDT\_DESC}, basta simplemente con cargarlo al registro \textit{gdtr}, habilitar el bit \textit{PE} en el CR0, y por último, setear el selector de segmento \textit{CS} con el índice (con permisos de supervisor) del descriptor de segmento de código de nivel 0. Esto último se hace mediante la instrucción "jmp INDICE\_RPL:saraza".\\

Ya estando en modo protegido, podemos setear el resto de los selectores de segmentos mediante la instrucción "mov". Aquí seleccionamos el selector de segmento fs con el índice de la GDT que tiene al descriptor del segmento de video.\\

Una vez hecho todo esto, escribir en la pantalla es una cuestión de usar el descriptor de video dentro de un ciclo de \textit{assembly}.... y de no pasarnos del limite del segmento (como comentamos en la introducción, el segmento de video es solo de 4KB, para cubrir la memoria de la pantalla solamente).