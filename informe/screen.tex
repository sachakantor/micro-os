Para mantener todas las funciones de manera más estructurada, lo primero que hicimos fue definir variables de precompilación que almacenen los colores de los fondos y letras de cada sector de la pantalla, así como de las tareas.\\

\subsection{pintar\_pantalla}
El comportamiento de esta función es, básicamente, inicializar las distintas areas de la memoria de video para que se muestre el mapa en la pantalla.\\
Describiendo en más detalle, lo que se hace es:
\begin{itemize}
 \item Pintamos la primera fila y colocamos el nombre del equipo en la esquina superior izquierda.
 \item Pintamos el fondo desde la segunda fila hasta que comienza el buffer de estados de las tareas.
 \item Mostramos los números horizontales y verticales de la arena.
 \item Colocamos los cuadrados negros de la arena, en dónde se van a mostrar las distintas partes de las tareas.
 \item Pintamos el fondo del header del último problema, sin ningún título.
 \item Pintamos el fondo del área dónde se va a mostrar el estado de ejecución del último problema.
 \item Mostramos los nombres de los registros del estado del último problema.
 \item Pintamos el fondo de los headers de las filas del buffer de estados de las tareas con sus respectivos colores.
 \item Pintamos el fondo del buffer de estados de las tareas.
 \item Pintamos la última fila de negro.
\end{itemize}

\subsection{pintar\_excepciones}
Esta función se encarga de mostrar el estado de ejecución del momento en el cuál se produjo una excepción.\\
\\
Como primer parámetro recibimos un puntero al primer caracter del mensaje de excepción que ocurrió. Utilizandoló se muestra en el header del 
área del último problema el mensaje de excepción arrojado y también se muestra en el buffer de estados de la tarea correspondiente.
Cabe aclarar que previo a mostrarlo en el header del último problema, se vacía el mismo, por si llegara a suceder que el error anterior fuese más largo.\\
\\
Como segundo parámetro toma el valor del registro esp luego de haber hecho pushad.
Por lo tanto, la pila a partir de la posición que recibe como parámetro, debe estar compuesta de la siguiente manera:\\
\begin{center}
  \begin{tabular*}{200mm}{l|c|}
  & \rule[0.25cm]{80mm}{0.01cm}\\
  esp & EDI\\
  esp + 4 & ESI\\
  esp + 8 & EBP\\
  esp + 12 & ESP(Previo al pushad)\\
  esp + 16 & EBX\\
  esp + 20 & EDX\\
  esp + 24 & ECX\\
  esp + 28 & EAX\\
  esp + 32 & ERRORCODE\\
  esp + 36 & EIP(Previo a la interrupción)\\
  esp + 40 & CS\\
  esp + 44 & EFLAGS\\
  esp + 48 & ESP(Previo a la interrupción)\\
  esp + 52 & SS\\
  esp + 56 & Dirección de retorno de la interrupción\\
  & \rule[-0.2cm]{80mm}{0.01cm}\\
  \end{tabular*}
\end{center}

Sabiendo que la pila está conformada como se describió previamente, accedemos al estado de cada registro y a los valores del stack y 
flags y los mostramos en formato hexadecimal en su lugar correspondiente en el área del último problema.\\
Un último detalle que cabe remarcar es que como los registros de segmento (ds, es, fs, gs) no se guardan en la pila, utliziamos assembler-inline 
para obtener el valor de los mismos. Esto lo podemos hacer ya que utilizamos segmentos flat y en todo el tiempo de ejecución no cambian.

\subsection{pintar\_posicion\_arena}
Lo que hacemos con esta función es actualizar la arena mostrada en pantalla, ya sea mostrando o borrando una página de código, de pila o compartida 
de la tarea que se está ejecutando (índice almacenado en la variable global tarea\_actual).\\
\\
Recibimos dos parámetros, un caracter y el índice (de 0 a 99) de la arena que debe modificarse.\\
Primero se calcula la fila y la columna del área de video del cuadrado correspondiente al índice recibido.
Notemos que la columna depende de si es una página compartida o es de código o pila, ya que si es compartida va en el cuadrado derecho y sino en el izquierdo.
A esta altura tomamos como columna la del cuadrado izquierdo.\\
\\
Según el caracter recibido por parámetro varía el comportamiento de nuestra función.\\
Si recibimos un caracter 'P' o 'C', simplemente se muestra en la posición ya calculada de la memoria de video un cuadrado con el color de la tarea a actualizar
con la letra recibida.\\
Si se recibe un caracter ' ', se muestra en la siguiente columna calculada de la memoria de video, un cuadrado con el color de la tarea actual, para mostrar
que se compartió esa página.\\
Si el caracter es 'X' lo que se hace es borrar (pintar del color de fondo nuevamente) el cuadrado que se calculó previamente de la memoria de video.\\
Por último si el caracter es 'S' y el cuadrado a la derecha del calculado tiene el color de fondo de la tarea actual borramos esa posción de la memoria de video (repintamos el color de fondo al incial).

\subsection{actualizar\_reloj\_tareas}
La idea de esta función es mover la manilla del reloj de la tarea actual a la siguiente posición.\\
Para esto, nos creamos un arreglo global que contiene la posición actual de la manilla de cada tarea. Lo que hacemos entonces es simplemente sumar 1 a ese arreglo y mostrar el caracter de reloj correspondiente a ese número de manera similar a lo hecho con el reloj de la tarea idle.
