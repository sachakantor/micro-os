Se puede verificar que la tarea se cargó, poniendo un breakpoint en la interrupción de reloj, y ejecutando 'info tss' una vez que se entró a dicha interrupción. Para confirmar que la tarea es la que nosotros queremos, se puede verificar el 'eip', 'cr3', 'tr', 'base' devueltos por la instrucción. 
Esto es valido porque al cargarse una tarea, el procesador levanta el contexto de la misma. Esto significa que pasa toda la información almacenada en la tss a los registros de contexto corresponientes, como ser los registros de propocito general, los de segmento, etc.