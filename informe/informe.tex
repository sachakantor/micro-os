\documentclass[10pt, a4paper]{article}
%Margenes de la pagina.  otra opcion, usar \usepackage{a4wide}
\usepackage[paper=a4paper, left=1.5cm, right=1.5cm, bottom=1.5cm, top=3.5cm]{geometry}
%este paquete permite incluir acentos.  Notar que espera un formato ANSI-blah de archivo.  Si en lugar de eso se tiene un utf8 (usual en los linux), entonces usar \usepackage[utf8]{inputenc}
\usepackage[utf8]{inputenc} 
%Este paquete es para que algunos titulos (como Tabla de Contenidos) esten en castellano
\usepackage[spanish]{babel}
%El siguiente paquete permite escribir la caratula facilmente
\usepackage{caratula}
%este paquete es innecesario para el TP, aca lo uso para recuadrar
\usepackage{framed}
\usepackage[pdftex]{graphicx}
\usepackage{amsmath}
\usepackage{amsfonts}
\usepackage{amssymb}
\usepackage{caption}
\usepackage{enumerate}
\usepackage{listings}
\usepackage{listing}
\usepackage{amsmath}
\lstset{language=C++, breaklines=true, showstringspaces=false, mathescape=false, fontadjust, tabsize=4}
\pagestyle{plain}
\newcommand{\real}{\ensuremath{\mathbb{R}}}


%Datos para la caratula
\materia{Organización del Computador II}



\titulo{Trabajo Práctico 3: 0x64\% LUCHA}


\grupo{Grupo: Marley (Sin huevos)}
\integrante{Gonzalo Lera Romero}{455/11}{gonzalo.lera.romero@gmail.com}
\integrante{Victor Wjugow}{174/11}{victorwjugow@gmail.com}
\integrante{Sacha Kantor}{009/11}{sachakantor@gmail.com}


\begin{document}
 
 %esto construye la caratula
	\maketitle
 	\tableofcontents
 	\newpage
 	\section {Introducción}
 		%\begin{minipage}{\linewidth}
			El desarrollo de un kernel que permita el funcionamiento correcto de una computadora no es una tarea fácil. El mismo es dependiente de la arquitectura que se esté utilizando y por esta razón implica un trabajo a muy bajo nivel.\\

La idea de este proyecto es la elaboración de un kernel (básico) que solucione los sigiuentes aspectos:
\begin{itemize}
	\item Administrar segmentos de datos y código.
	\item Administrar la memoria principal utilizando el mecanismo de paginación.
	\item Capturar cualquier tipo de excepción que pueda generar un programa.
	\item Permitir la ejecución de programas en dos niveles (Usuario y Supervisor).
\end{itemize}

A lo largo de este TP, hicimos cambios en los archivos y tomamos decisiones sobre como configurar las distintas estructuras de datos que usa la arquitectura de intel para el manejo de memoria (segmentación y paginación) y tareas (\textit{tss}).\\

A continuación daremos un resúmen de esas desiciones y cambios.

\subsection{Archivos nuevos y modificados}
\begin{itemize}
	\item \textit{defines.h}: Se modificaron los defines de las pilas para que apunten a la base en lugar e al tope.\\
		Se agregó el define TAM\_TSSS para usar en el \textit{array} de tareas (ya que decidimos incluir a la tarea idle e inicial en él).\\
		Definimos variables de precompilación para los selectores de descriptores de segmento y de tareas.
	\item \textit{macrosmodoprotegido.mac}: Cambiamos unas etiquetas ya que entraban en conflicto con etiquetas de \textit{macrosmodoreal.mac} si se incluían ambos archivos en el \textit{kernel.asm}.
	\item \textit{mensajesInt.mac}: Son los mensajes de interrupción que luego usará \textit{isr.asm}.
	\item \textit{selectores.mac}: Son los mismos selectores definidos en el \textit{defines.h} pero para nasm.
	\item \textit{paginación.h/.c}: Son las estructuras que usamos para configurar el directorio y tablas de páginas del \textit{kernel}, y que son usadas por la MMU para configurar el de las tareas. También provee una función para el directorio y tablas de páginas del \textit{kernel}.
	\item \textit{tss.h/c}: Aquí solo decidimos incluir las \textit{TSS} de las tareas inicial e idle en el \textit{array} global \textit{tsss[]}
	\item \textit{*.hpp}: Archivos provistos para poder "hacer un loop" de una invocación a una macro. Ver la sección "Macro Vudú" para más información.
\end{itemize}

\subsection{GDT}
Nuestra GDT se ordena de la siguiente manera:\\

\begin{tabular}{|l|l|l|l|l|}
	\hline
	Tipo & Descripción  & Índice & Nivel& Índice+RTL (Selector) \\ \hline
	Nulo & N/A          & 0      & N/A  & N/A  \\ \hline
	TSS  & Tarea 1      & 1      & 3    & 0xB  \\ \hline
	TSS  & Tarea 2      & 2      & 3    & 0x13 \\ \hline
	TSS  & Tarea 3      & 3      & 3    & 0x1B \\ \hline
	TSS  & Tarea 4      & 4      & 3    & 0x23 \\ \hline
	TSS  & Tarea 5      & 5      & 3    & 0x2B \\ \hline
	TSS  & Tarea Inicial& 6      & 0    & 0x30 \\ \hline
	TSS  & Tarea Idle   & 7      & 0    & 0x38 \\ \hline
	Nulo & N/A          & 8      & N/A  & N/A\\ \hline
	Nulo & N/A          & 9      & N/A  & N/A\\ \hline
	GDT  & Codigo       & 10     & 0    & 0x50 \\ \hline
	GDT  & Datos        & 11     & 0    & 0x58 \\ \hline
	GDT  & Codigo       & 12     & 3    & 0x63 \\ \hline
	GDT  & Datos        & 13     & 3    & 0x6B \\ \hline
	GDT  & Video        & 14     & 0    & 0x70 \\ \hline
\end{tabular}

\subsection{TSS}
Nuestro \textit{array tsss[]} que contiene las \textit{TSS} de todas las tareas se describe en la siguiente tabla:\\

\begin{tabular}{|l|l|l|l|l|l|l|l|l|}
	\hline
	Indice & Tarea  & SS0:ESP0 & CR3     & EIP      & SS:ESP        & CS   & ES/DS/GS & FS \\ \hline
	0      & 1      & 0x58:STT1& 0x30000 & 0x3A0000 & 0x6B:0x3B1000 & 0x63 & 0x6B     & 0x6B\\ \hline
	1      & 2      & 0x58:STT2& 0x31000 & 0x3A0000 & 0x6B:0x3B1000 & 0x63 & 0x6B     & 0x6B\\ \hline
	2      & 3      & 0x58:STT3& 0x32000 & 0x3A0000 & 0x6B:0x3B1000 & 0x63 & 0x6B     & 0x6B\\ \hline
	3      & 4      & 0x58:STT4& 0x33000 & 0x3A0000 & 0x6B:0x3B1000 & 0x63 & 0x6B     & 0x6B\\ \hline
	4      & 5      & 0x58:STT5& 0x34000 & 0x3A0000 & 0x6B:0x3B1000 & 0x63 & 0x6B     & 0x6B\\ \hline
	5      & Inicial& N/A      & N/A     & N/A      & N/A           & N/A  & N/A      & N/A\\ \hline
	6      & Idle   & N/A      & 0x21000 & 0x3A0000 & 0x58:0x40000  & 0x50 & 0x58     & 0x70\\ \hline
\end{tabular}
\\
La tarea Inicial no requiere de datos pues solo se usa para tener algún lugar donde guardar el contexto a cambiar a la tarea idle. Luego de eso, el contexto inicial nunca más se volverá a usar.\\

\subsection{Macro Vudu}
En los archivos \textit{isr.h} e \textit{idt.c} se puede observar la utilización de la macro \textit{BOOST\_PP\_REPEAT}, la misma está provista por una suite de archivos que bajamos de internet (en el codigo se cita a la fuente) que nos permite ahorranos invocar a una misma macro muchas veces con distintos parametros (de alguna manera, simula un \textit{loop} a nivel de macros de C).

\subsection{Funcionalidad Task-by-Task}
En el marco del TP, y por motivos de comodidad a la hora de las pruebas, decidimos agregar una nueva funcionalidad al teclado. Cuando el combate entre tareas se encuentra pausad, es posible presionar la tecla 's' para ejecutar la siguiente tarea por 1 solo tick. Presionar la 's' cuando el combante no se encuentra pausado no tiene ningún efecto (salvo la atención de interrupción del teclado). 		
 		%\end{minipage}
	  	\pagebreak
	\section {Screen}
	    Para mantener todas las funciones de manera más estructurada, lo primero que hicimos fue definir variables de precompilación que almacenen los colores de los fondos y letras de cada sector de la pantalla, así como de las tareas.\\

\subsection{pintar\_pantalla}
El comportamiento de esta función es, básicamente, inicializar las distintas areas de la memoria de video para que se muestre el mapa en la pantalla.\\
Describiendo en más detalle, lo que se hace es:
\begin{itemize}
 \item Pintamos la primera fila y colocamos el nombre del equipo en la esquina superior izquierda.
 \item Pintamos el fondo desde la segunda fila hasta que comienza el buffer de estados de las tareas.
 \item Mostramos los números horizontales y verticales de la arena.
 \item Colocamos los cuadrados negros de la arena, en dónde se van a mostrar las distintas partes de las tareas.
 \item Pintamos el fondo del header del último problema, sin ningún título.
 \item Pintamos el fondo del área dónde se va a mostrar el estado de ejecución del último problema.
 \item Mostramos los nombres de los registros del estado del último problema.
 \item Pintamos el fondo de los headers de las filas del buffer de estados de las tareas con sus respectivos colores.
 \item Pintamos el fondo del buffer de estados de las tareas.
 \item Pintamos la última fila de negro.
\end{itemize}

\subsection{pintar\_excepciones}
Esta función se encarga de mostrar el estado de ejecución del momento en el cuál se produjo una excepción.\\
\\
Como primer parámetro recibimos un puntero al primer caracter del mensaje de excepción que ocurrió. Utilizandoló se muestra en el header del 
área del último problema el mensaje de excepción arrojado y también se muestra en el buffer de estados de la tarea correspondiente.
Cabe aclarar que previo a mostrarlo en el header del último problema, se vacía el mismo, por si llegara a suceder que el error anterior fuese más largo.\\
\\
Como segundo parámetro toma el valor del registro esp luego de haber hecho pushad.
Por lo tanto, la pila a partir de la posición que recibe como parámetro, debe estar compuesta de la siguiente manera:\\
\begin{center}
  \begin{tabular*}{200mm}{l|c|}
  & \rule[0.25cm]{80mm}{0.01cm}\\
  esp & EDI\\
  esp + 4 & ESI\\
  esp + 8 & EBP\\
  esp + 12 & ESP(Previo al pushad)\\
  esp + 16 & EBX\\
  esp + 20 & EDX\\
  esp + 24 & ECX\\
  esp + 28 & EAX\\
  esp + 32 & ERRORCODE\\
  esp + 36 & EIP(Previo a la interrupción)\\
  esp + 40 & CS\\
  esp + 44 & EFLAGS\\
  esp + 48 & ESP(Previo a la interrupción)\\
  esp + 52 & SS\\
  esp + 56 & Dirección de retorno de la interrupción\\
  & \rule[-0.2cm]{80mm}{0.01cm}\\
  \end{tabular*}
\end{center}

Sabiendo que la pila está conformada como se describió previamente, accedemos al estado de cada registro y a los valores del stack y 
flags y los mostramos en formato hexadecimal en su lugar correspondiente en el área del último problema.\\
Un último detalle que cabe remarcar es que como los registros de segmento (ds, es, fs, gs) no se guardan en la pila, utliziamos assembler-inline 
para obtener el valor de los mismos. Esto lo podemos hacer ya que utilizamos segmentos flat y en todo el tiempo de ejecución no cambian.

\subsection{pintar\_posicion\_arena}
Lo que hacemos con esta función es actualizar la arena mostrada en pantalla, ya sea mostrando o borrando una página de código, de pila o compartida 
de la tarea que se está ejecutando (índice almacenado en la variable global tarea\_actual).\\
\\
Recibimos dos parámetros, un caracter y el índice (de 0 a 99) de la arena que debe modificarse.\\
Primero se calcula la fila y la columna del área de video del cuadrado correspondiente al índice recibido.
Notemos que la columna depende de si es una página compartida o es de código o pila, ya que si es compartida va en el cuadrado derecho y sino en el izquierdo.
A esta altura tomamos como columna la del cuadrado izquierdo.\\
\\
Según el caracter recibido por parámetro varía el comportamiento de nuestra función.\\
Si recibimos un caracter 'P' o 'C', simplemente se muestra en la posición ya calculada de la memoria de video un cuadrado con el color de la tarea a actualizar
con la letra recibida.\\
Si se recibe un caracter ' ', se muestra en la siguiente columna calculada de la memoria de video, un cuadrado con el color de la tarea actual, para mostrar
que se compartió esa página.\\
Si el caracter es 'X' lo que se hace es borrar (pintar del color de fondo nuevamente) el cuadrado que se calculó previamente de la memoria de video.\\
Por último si el caracter es 'S' y el cuadrado a la derecha del calculado tiene el color de fondo de la tarea actual borramos esa posción de la memoria de video (repintamos el color de fondo al incial).

\subsection{actualizar\_reloj\_tareas}
La idea de esta función es mover la manilla del reloj de la tarea actual a la siguiente posición.\\
Para esto, nos creamos un arreglo global que contiene la posición actual de la manilla de cada tarea. Lo que hacemos entonces es simplemente sumar 1 a ese arreglo y mostrar el caracter de reloj correspondiente a ese número de manera similar a lo hecho con el reloj de la tarea idle.


	\section {Desarrollo}
	  	\subsection{Ejercicio 1}
	  		\begin{minipage}{\linewidth}
	  			Aprovechando los \textit{structs} \textit{gdt\_entry} y \textit{gdt\_descriptor} provistos por la cátedra en el archivo \textit{gdt.h}, la tarea de completar la GDT con 5 segmentos de datos no conllevó una gran dificultad. Dichos \textit{structs} nos facilitan la configuración de los distintos bits de las estructuras de manejo de memoria utilizadas por la arquitectura de intel, ya que podemos trabajar con los nombres de los campos de estas estructuras olvidandonos de que bits especificos se deben setear.

Siguiendo con nuestras decisiones comentadas durante la introducción, en el archivo \textit{gdt.c} nos encontramos con un \textit{array} global (declarado fuera de cualquier funcion, es decir, sus datos serán escritos en el binario del \text{kernel}) de descriptores de segmentos llamado \textit{gdt}. En el mismo, inicializamos los primeros 10 elementos con descriptores nulos, y luego incializamos los 5 segmentos pedidos por el enunciado del ejercicio (codigo, datos -para supervisor y usuario- y video -solo para supervisor). Para más detalles sobre los datos de los descriptores, y sus índices se enunciados en la introducción.\\

Luego nos encontramos con otra variable global, \textit{GDT\_DESC}, que es del tipo \textit{gdt\_descriptor}. El mismo se inicializa con el tamaño del \text{array gdt} y la dirección de memoria del mismo (básicamente, la información que debe tener el registro \textit{gdtr} para pasar a modo protegido).\\

Luego, ya en el \textit{kernel.asm}, debemos empezar el proceso de pasar a modo protegido. Al ya tener en el binario del \textit{kernel} la estructura de la GDT (en el \textit{array} que configuramos arriba), y también tener el selector de la GDT en \textit{GDT\_DESC}, basta simplemente con cargarlo al registro \textit{gdtr}, habilitar el bit \textit{PE} en el CR0, y por último, setear el selector de segmento \textit{CS} con el índice (con permisos de supervisor) del descriptor de segmento de código de nivel 0. Esto último se hace mediante la instrucción "jmp INDICE\_RPL:saraza".\\

Ya estando en modo protegido, podemos setear el resto de los selectores de segmentos mediante la instrucción "mov". Aquí seleccionamos el selector de segmento fs con el índice de la GDT que tiene al descriptor del segmento de video.\\

Una vez hecho todo esto, escribir en la pantalla es una cuestión de usar el descriptor de video dentro de un ciclo de \textit{assembly}.... y de no pasarnos del limite del segmento (como comentamos en la introducción, el segmento de video es solo de 4KB, para cubrir la memoria de la pantalla solamente).
	  		\end{minipage}
	  	
	  	\subsection{Ejercicio 2}
	  		\begin{minipage}{\linewidth}
	  			\subsubsection*{Punto a}
Lo que debiamos hacer en este ejercicio, era completar las primeras 20 entradas de la idt, que se corresponden con las interrupciones que vienen especificadas por Intel.
Para esto, utilizamos varias macros. En el archivo idt.c se invoca a la macro BOOST\_PP\_REPEAT, que lo que hace es invocar 20 veces a la macro IDT\_ENTRY, definida más arriba.\\

En IDT\_ENTRY se indica como se completa una entrada de la IDT. En los campos $offset$ ponemos la dirección de memoria en donde se encuentra la rutina de atención de interrupcion correspondiente (\_isrX es el nombre de la rutina). En el campo $attr$ ponemos el valor $0x8E00$ que setea el bit $P$, pone $DPL$ en nivel 0, y setea el bit $D$. También setea otros bits que Intel requiere para que esta entrada sea considerada como una 'Interrupt Gate'. Por último seteamos el campo $segsel$ para que se utilice el segmento de Codigo de nivel 0.\\

Las rutinas \_isrX se definen en el archivo isr.asm. Una rutina se define con la macro ISR. A su vez, esta macro se llama 20 veces desde un loop del nasm, cuyo código esta justo debajo de la macro ISR. 
En la macro ISR, está comentada la línea que imprimía por pantalla la excepción que lanzó la interrupción, como se pedía en este ejercicio. En la línea comentada se ve que se llama a la macro provista por la cátedra, 'IMPRIMIR\_TEXTO'. Como parámetros le pasabamos el mensaje, la longitud, color, linea y columna. El mensaje y la longitud se definen en el archivo 'mensajesInt.mac' mediante otra macro 'MENSAJE'.

\subsubsection*{Punto b}
Para que el procesador utilice la IDT, invocamos, desde kernel.asm, a la función inicializar\_idt. La misma esta definida en idt.h e implementada en idt.c. Dentro de esta función está el código comentado en el punto anterior. Luego, ejecutamos la instrucción 'lidt [IDT\_DESC]' que carga la variable IDT\_DESC en el registro 'IDTR'. Lo que se carga, es el puntero al vector de rutinas de atención de interrupcines y el tamaño de este vector, tal cual lo requiere el formato del registro.
La variable IDT\_DESC está definida al final de idt.c\\
Para probar el funcionamiento de interrupciones, tenemos comentado el código del ejercicio 1, que lanza una excepción. Además se tenía un 'sti' que se movió para ser ejecutado luego de activar el pic.
	  		\end{minipage}
	  	
	  	\subsection{Ejercicio 3}
	  		\begin{minipage}{\linewidth}
	  			Lo que hicimos para resolver el problema planteado por este ejercicio fue una función en C llamada \textit{inicializar\_dir\_kernel} en el archivo \textit{paginacion.c}.\\

Previo a comenzar, creamos una función llamada \textit{set\_dir\_entry} que recibe los parámetros que han de setearse en una dir\_entry, un puntero a la posición dónde comienza el directorio y el índice de la entrada que debe crear la entrada y en base a esto la construye.\\
Análogamente, creamos otra función llamada \textit{set\_tbl\_entry} que recibe los parámetros que pueden setearse en una table\_entry, un puntero a la base de la tabla y el índice de la entrada que se quiere crear y también la construye.\\

Lo primero que hacemos para inicializar la paginación del kernel es posicionar un puntero en la posición 0x21000 para comenzar a escribir la primera entrada de su directorio. Luego la creamos utilizando la función \textit{set\_dir\_entry} de la siguiente manera:\\
\\\
\begin{tabular}{|l|l|l|l|l|l|l|l|l|l|l|}
	\hline
	Base & Disponible & G & PS & 0 & A & PCD & PWT & U/S & R/W & P\\
	\hline
	0x22 & 0 & 0 & 0 & 0 & 0 & 0 & 0 & 0 & 1 & 1\\
	\hline
\end{tabular}
\\ \\\
Analizando un poco, la base está en 0x22 ya que debe describir la base de la tabla de páginas del kernel y como el sistema de paginación asume que esta alineada a 4K la convierte en la dirección 0x22000 que efectivamente es lo que queríamos. \\
El bit U/S está en 0 ya que queremos que el kernel tenga privilegios de sistema para poder acceder a los primeros dos MB, tambien queremos que pueda leer y escribir cualquier posición (R/W = 1) y aclarar que la tabla de páginas está presente en la memoria (P = 1).\\
\\\
Ya habiendo seteado la primera entrada del directorio del kernel (la única que vamos a utilizar), ponemos las demás entradas del directorio en 0 (para que P = 0) y las ignore.\\
Lo que sigue es consturir la tabla de páginas del kernel, para esto se nos pidió hacer identity mapping sobre los primeros 2MB. Esto quiere decir mapear las primeras 512 entradas de la tabla a las primeras 512 páginas de la memoria (512 páginas * 4KB = 2MB). El resto de la memoria lo dejamos en no presente.\\
Para esto, utilizando \textit{set\_tbl\_entry}, creamos las primeras 512 entradas de la siguiente manera:\\
\\\
\begin{tabular}{|l|l|l|l|l|l|l|l|l|l|l|}
	\hline
	Base & Disponible & G & PAT & D & A & PCD & PWT & U/S & R/W & P\\
	\hline
	i & 0 & 0 & 0 & 0 & 0 & 0 & 0 & 0 & 1 & 1\\
	\hline
\end{tabular}
\\ \\\
Esto lo hacemos ya que para hacer identity mapping, la primera entrada va a tener un 0 en la base, y va a apuntar a la dirección 0 de la memoria, que es la base de la primera página. La segunda entrada va a tener un 1 y como la unidad de paginación asume que las páginas son de 4KB entonces la dirección base que queda es 0x1000 que es justamente la dirección de la página 2 de la memoria y así sucesivamente.\\
Por otro lado el kernel tiene que poder leer y escribir (R/W = 1) y acceder con privilegios de sistema (U/S = 0).\\
El resto de las entradas las ponemos en 0 (en particular P = 0), ya que no queremos direccionar más allá de 2MB.\\\
\\
Por último en esta función mapeamos la dirección virtual 0x3A0000 a la 0x10000 que es dónde va a estar el código de la tarea idle, para que este pueda ejecutarse asumiendo que esa es su posición en memoria (recordemos que la tarea idle usa el directorio de páginas del kernel).\\
\\\
Para habilitar paginación simplemente hay que poner el bit mas significativo del CR0 en 1, lo cuál hicimos desde assembler con la instrucción OR.\\\
Nota: No fue necesario hacer un tlbflush() luego de completar el mapeo de páginas del kernel porque todavía no estaba activada la paginación.
	  		\end{minipage}
	  	
	  	\subsection{Ejercicio 4}
	  		\begin{minipage}{\linewidth}
	  			Primero vamos a explicar el comportamiento de las funciones \textit{mapear\_pagina} y \textit{unmapear\_pagina}.\\
Mapear una dirección virtual a una física quiere decir poner las entradas del directorio y de las tablas de páginas de determinada manera para que al intentar acceder a una dirección virtual x se llegue a la dirección fisica especificada.\\
Para hacer esto debemos recordar que significa la dirección virtual. La misma está compuesta de la siguiente manera:
\\ \\\
\begin{tabular}{|l|l|l|}
\hline
\textbf{Índice del directorio} & \textbf{Índice de la tabla de página} & \textbf{Offset dentro de la página}\\
\hline
\end{tabular}
\\ \\\
La unidad de paginación busca la base del directorio en el CR3, luego busca la base de la tabla de página en la entrada especificada por el \textbf{Índice del directorio}, luego busca la base de la página especificada por la entrada \textbf{Índice de la tabla de página} y por último le suma el \textbf{Offset} y asi accede al byte requerido.\\
Lo que nosotros hacemos es justamente setear todo el camino que hace la unidad, con entradas presentes y que apunten a donde deben, para que finalmente la base que encuentre en la tabla de páginas sea tal que sumada con el offset de la dirección física que nos pidieron.\\\ \\

En otras palabras, primero con el CR3 buscamos la base del directorio, luego vamos a la entrada especificada y obtenemos la tabla de páginas que nos indica dicha entrada. Luego con la tabla de páginas vamos a la entrada especificada por la dirección virtual y le cambiamos la base a la primera dirección física menor alineada a 4KB (o sea comienzo de página).\\\ \\
Para unmapear una página lo que hacemos es lo mismo, pero en vez de cambiar la base de la entrada de la tabla de páginas, simplemente ponemos que esa entrada ya no está presente en la memoria.\\
\\
Nota: No necesitamos invocar a la función tlbflush dentro de las funciones 'unmapearpagina' y 'mapearpagina', ya que cuando una tarea pide mapear una página, despues de hacerlo se le sede el quantum que quede a la tarea Idle, con lo que no intentará acceder a dicha página de momento. Y cuando la tarea que pidió la página, es puesta a correr nuevamente, previamente tuvo que haber habido un cambio de contexto, y cuando lo hay, el procesador limpia el buffer de tlb.
\\ \\\
Ya habiendo explicado como mapear páginas, veamos como funciona \textit{inicializar\_dir\_tarea}. La idea de esta función es incializar los directorios y tablas de páginas de todas las tareas, luego copiamos el código y la pila de cada tarea a la arena y por último mapeamos su código y pila a unas direcciones virtuales identicas para cada tarea.\\
\\
Analizando las direcciones que iban a ser accedidas por las tareas nos dimos cuenta que se necesita hacer identity mapping sobre los primeros 2MB con privilegio de sistema ya que solamente queremos que el kernel pueda acceder en caso de que ocurra una interrupción. Y después el resto debe estar no presente.\\
Para hacer esto hacemos identity mapping de manera similar al ejercicio anterior.\\
Luego mapeamos por cada tarea, la dirección virtual 0x3A0000 a la dirección física dónde va a copiarse el código de la tarea dentro de la arena y la dirección virtual 0x3B0000 a la dirección física dónde estará la pila. Esto lo hacemos utilizando la función \textit{mapear\_pagina} mencionada previamente.\\

Lo siguiente que hacemos es actualizar la pantalla mostrando dónde están las páginas de código y pila de las tareas en la arena llamando a la función \textit{pintar\_posicion\_arena}. Por último copiamos todas las tareas (código y pila) de sus posiciones inciales dentro de los primeros 2MB a las posiciones generadas en el archivo \textit{rand.c} dentro de la arena.

	  		\end{minipage}
	  		
	  	\subsection{Ejercicio 5}
	  		\begin{minipage}{\linewidth}
	  			\subsubsection* {Punto a}
Las 3 entradas que se requieren agregar, son agregadas en el archivo idt.c. Para esto copiamos la definición de la macro 'IDT\_ENTRY', utilizada en el ejercicio 2. Para las interrupciones de reloj y teclado, no se modificó nada más que el campo $offset$ al copiar la macro, para apunten a las rutinas correspondientes. En cambio para la rutina '0x45' también se modificó el campo $attr$, bajando el RPL a nivel 3 (Usuario) para que esta interrupción pueda ser invocada por tareas que corran en dicho nivel.

\subsubsection* {Punto b}
La rutina asociada a la interruoción de reloj se definió en el archivo isr.asm. La misma se llama 'reloj'. Ahí se encuentra comentado el código, tal cual se implementó al momento de realizar este ejercicio. Lo único que hacía era llamar a la función provista por la cátedra, que actualizaba el reloj en la esquina inferior derecha de la pantalla.


\subsubsection* {Punto c}
La rutina del teclado, se encuentra en el mismo archivo, y se llama 'teclado'. La misma, además de cumplir con lo pedido del ejercicio, resuelve el requerimiento de botón Pausa, Restore, y un agregado nuestro, 'Step By Step'. \\

En la rutina, se le notifica al pic que ya atendimos la interrupción, se lee el puerto 0x60 para obtener el makecode de la tecla presionada, y luego se comineza a identificar que tecla fue la presionada. En caso de haber sido 'P' seteamos la variable global 'pause', definida en 'sched.c', en 1. Si fue 'R' volvemos la variable 'pause' a 0, y si fue 'S', seteamos la variable 'step\_by\_step' en 1. El comportamiento que activa al estar activado el modo 'step by step' es que, se ejecutará una tarea solamente, y luego se volverá a la tarea IDLE, como si se estuviese en modo Pausa.\\

Finalmente, si la tecla presionada fue un número, se le suma $0x2F$ al makecode para convertirlo a ASCII. Se le suma a la variable 'aleatorio' lo obtenido, para generar un formato cualquiera. Al formato se lo manipula para forzar que el color de letra sea blanca, y no titile. Por último se mueve el codigo ASCII y el formato a la última posición de la primer fila de la pantalla.

\subsubsection* {Punto d}
La rutina de la interrupción 0x45 se llama 'serviciosOS'. El código correspondiente a este ejercicio se encuentra comentado. Solo avisaba al pic que se atendió la interrupción y seteaba eax en 42.
	  		\end{minipage}
	  	
	  	\subsection{Ejercicio 6}
	  		\begin{minipage}{\linewidth}
	  			Como ya menciona anteriormente, en la sección de la intruducción comentamos nuestras decisiones en cuanto a las estructuras de datos que utiliza la arquitectura de intel para el manejo de memoria y tareas.\\

En este ejercicio, lo primero que hicimos fue trabajar con el \textit{array} global de \textit{TSS} -llamado \textit{tsss}- que se encuentra en el archivo \textit{tss.c}. El mismo hace uso del \textit{struct tss} que se encuentra definido en \textit{tss.h}, que fue provisto por la cátedra. Este \textit{array} contendra las \textit{TSS} de todas las tareas, incluyendo a las tareas \textit{idle} e \textit{inicial}. Básicamente, siguiendo las decisiones que describimos durante la introducción, completamos los datos de las 7 \textit{TSS} con los datos correspondientes a cada tarea.\\

A su vez, volvimos al archivo \textit{gdt.c} donde se encuentra otro \textit{array} global, el \textit{array} de la GDT. Aquí, lo que hicios fue tomar las entradas 1-7 para utilizarlas como descriptores de \textit{TSS}. En estos descriptores, pudimos volcar toda la información correspondiente a cada tarea, excepto por la dirección base de las \textit{TSS}, ya que estas están definidas en el archivo \textit{tss.c}, y hasta que no se compile, no se puede saber sus direcciones de memoria.\\

Para poder completar los descriptores de \textit{TSS} en la GDT con las direcciones bases de las \textit{TSS}, solo nos queda resolver el problema en tiempo de ejecución (por lo ya explicado en el párrafo anterior). Para ello, invocamos desde el \textit{kernel.asm} a la función \textit{inicializar\_tsss} que está definida en el \textit{sched.c}. Esta función recorre todos los elementos del \textit{array tsss}, y utiliza la dirección de memoria de los mismos para setear los campos faltantes en el descriptor correspondiente en la GDT.\\

Ya aquí, sólo nos falta saltar a la tarea \textit{IDLE}. Como sabemos cual es el índice del descriptor de tarea que queremos utilizar (pues, no solo hemos comentado cual era en la introducción de este informe, sino que además lo acabamos de configurar), con utilizar la instrucción "jmp INDICE\_TAREA\_IDLE:saraza" estaremos comenzando todo el procesos de la arquitectura de intercambio de tarea. Previamente, hay que cargar en el registro \textit{tr} el índice del descriptor de tarea de la tarea inicial, para que al comenzar el proceso de intercambio de tarea, el contexto actual tenga una \textit{TSS} valida donde poder guardar la información, a pesar de que no volveremos a conmutar hacia ella nunca más.\\

También vale la pena aclarar, que estos "índices" de descriptores de tareas han de tener la \textit{RTL} correspondiente al nivel de privilegios con el que se correrá la tarea. Para más detalles, se puede ver la sección de introducción de este documento.
	  		\end{minipage}
	  	
	  	\pagebreak
	  	\subsection{Ejercicio 7}
	  		%\begin{minipage}{\linewidth}
	  			Para comenzar, primero tuvimos que decidir como íbamos a admistrar las tareas. Es decir, como y donde guardamos los datos que nos permiten saber cual es la tarea actual y cual es la siguiente tarea "viva" a la que se a de permutar. También esto está estrechamente relacionado sobre como manejaríamos luego las funcionalidades de \textit{pause}, \textit{restart} y \textit{tast-by-task} (esta última explicada en la introducción del informe).\\

Decidimos entonces tener 4 variables globales:
\begin{itemize}
    \item \textit{tareas[]} que es un \textit{array} que tendrá los selectores de descriptores de tareas de cada tarea que correrá con privilegios de usuario (con los "RPL" adecuados). Habrá una relación entre la posición del selector en el \textit{array} y la tarea apuntada, en nuestro caso \textit{tareas[i]} tendrá el selector de la tarea \textit{i+1}.
    \item \textit{indiceTareaActual} nos indica cual es la tarea cuyo contexto se encuentra cargado.
    \item \textit{pause} actuará como un valor \textit{booleano} para indicarnos que la tecla 'p' del teclado ha sido presionada.
    \item \textit{step\_by\_step} tendrá la misma funcionalidad que \textit{pause}, excepto que para la tecla 's'
\end{itemize}

Todas estas variables se encuentran declaradas en el archivo \textit{sched.c}.\\

Ahora si, habiendo tomado estas decisiones, creamos una función cuya finalidad es inicilizar estos valores, la misma es \textit{inicializar\_scheduler}, y básicamente lo que hace es generar los selectores de descriptores de tareas dentro del array (manteniendo la relación mencionada más arriba) e inicializar \textit{pause} en 1 (para las tareas no empiecen a matarse al instante, sino que haya que presionar la tecla 's'. Esta funcionalidad se implementa luego), \textit{step\_by\_step} en 0 y \textit{indiceTareaActual} en 4 (esto último debido a que la función \textit{próximo\_índice} -implementada para este ejercicio- nos debería devolver el índice de la siguiente tarea, que sería 0, y queremos que eso ocurra justamente para que la tarea 1 sea la primera al comenzar la batalla).\\

Luego implementamos la función \textit{próximo\_índice}, la misma tendra una interacción muy fuerte con la rutina de atención de interrupciones del reloj y del teclado, ya que ambas se sostienen en esta función. El comportamiento de esta función, sin entrar en detalles muy técnicos, es el de retornar el selector de descriptor de \textit{TSS} correspondiente a la próxima tarea a ejecutarse y actualizar la variable global \textit{indiceTareaActual}. En caso de que no haya más tareas vivas, o el \textit{flag} de \textit{pause} este en "prendido" (es decir, que se haya presionado la tecla 'p' -tarea de la rutina de interrupción del teclado-), la función retorna el selector de la tarea \textit{IDLE}. En el caso de estar "pausados", si el \textit{flag} de \textit{step\_by\_step} es encendido, retornará el índice correspondiente a la tarea siguiente (que bien podría ser \textit{IDLE}, si no hay tareas vivas).\\

Como mencionamos, la funcionalidad de intercambio de tareas se apoya básicamente en 3 cosas: las rutinas de atención de interrupción del teclado y reloj, y de la función \textit{próximo\_índice}. Ésta última la hemos explicado, falta ver como interactúa con las interrupciones.\\

La rutina del reloj, a la hora de implementar la funcionalidad del \textit{scheduler}, básicamente lo que ha de hacer es llamar a la función \textit{próximo\_índice} y saltar a la tarea correspondiente (recordemos que esta funcion ya nos devuelve el selector del descriptor de la tarea en la GDT con la "RPL" adecuada). Tristemente, hay algunas complejidades que debemos de tener en cuenta, principalmente el hecho de que hacer un "jump" a una tarea que es la actual generaría una excepción, ya que su descriptor de \textit{TSS} va a estar marcado como \textit{busy}. Para ello, en el código \textit{asm}, realizamos comparaciones contra el registro \textit{tr} actual y realizamos la conmutación de tareas si y sólo si realmente se cambia de tarea, en caso contrario, continuamos con la ejecución del contexto actual.\\

A su vez, en la rutina del reloj es donde nos encargamos de invocar a la función \textit{actualizar\_reloj\_tareas} de manera tal de actualizar el reloj de la tarea a la que se va a conmutar.\\

En cuanto a la interrupción del teclado, simplemente agregamos al chequeo que ya habíamos hecho durante el ejercicio 5 la funcionalidad de reconocer los \textit{breakcodes} de las teclas 'p', 'r' y 's'. En el caso de la 'p', seteamos la variable global \textit{pause} en 1, y si era la 'r' la seteamos en 0. En el caso de la 's', seteamos la variable \textit{step\_by\_step} en 1 (luego la función \textit{próximo\_índice} se encargara de ponerla en 0).\\

A estas alturas ya tenemos implementado el intercambio de tareas según el reloj y el teclado. Nos faltaría hablar sobre la implementación de los servicios del \textit{kernel} y el manejo de excepciones que pueden producir las tareas.\\

Cualquier tarea, al realizar alguna operación no valida, hará que el procesador genera una excepción. En tal caso, nuestro "sistema operativo" (a.k.a. \textit{referí}) ha de eliminar a la tarea, liberar los recursos que tenía asignada y comentar cual fue el error por la pantalla. Para todas estas acciones, nos hemos apoyado en las rutinas de atención de interrupciones y en la función \textit{chuckNorris} (que mata el 100\% de gérmenes y bacterias, no como otros productos), que se encuentra implementada en el archivo \textit{sched.c}.\\

Para comenzar, lo primero que hacemos es usar macros de C y Nasm en los archivos \textit{idt.c} e \textit{isr.asm} correspondientemente. En el primero, simplemente hacemos que todas las interrupciones de la IDT apunten a una posición de memoria que se genera en el \textit{isr.asm}, y en este utilizamos un loop de Nasm para generar esas posiciones de memoria y su rutina asociada (que es la misma). Esto lo hacemos para las primeras 20 excepciones de intel. De la excepción número 0x45 y 0x46 hablaremos luego.\\

La rutina que genera el \textit{loop} del nasm, mencionada en el párrafo anterior, simplemente lo que hace es invocar a \textit{chuckNorris} (que viene porque quiere, no porque lo invoquen; a no confundir) con los parametros correspondientes.\\

La implementación de \textit{chukNorris}, lo que hace es marcar a la tarea como "muerta" de manera tal de que \textit{próximo\_índice} no la tome más en cuenta, y luego actualiza la pantalla mediante funciones de \textit{screen.c}. Esta "actualización" consiste en borrar las páginas de memoria que la tarea tenía asignadas, imprimir el mensaje de la interrupción que accionó la tarea antes de ser eliminada e imprimir el estado de los registros que tenía la tarea antes de expirar.\\

Por último, nos falta hablar de los servicios del sistema operativo y de la interrupción 0x46. Los servicios a implementar por el TP nos pedían que le permitiésemos a las tareas pedír una página de memoria de la arena y también pasarles la información de cuales éran los índices en la arena de sus páginas de código y datos actuales. Ambos servicios podían ser invocados mediante la interrupción número 0x45. Para implementarlas, creamos dos funciones: \textit{set\_page} y \textit{get\_code\_stack}.\\

\textit{get\_code\_stack} toma simplemente el registro \textit{CR3}, que en el contexto corresponde a la tarea actual (ya que las interrupciones/excepciones no accionan un cambio de contexto) y en conjunto con las direcciones virtuales TASK\_CODE y TASK\_STACK (que son las direcciones virtuales de todas las tareas) calculan cual es el índice en la arena de las páginas de la tarea. Luego devuelve estos valores mediante los registros \textit{ecx} y \textit{ebx}.\\

\textit{set\_page} chequea, antes que nada, si el índice que la tarea pidio que se le mapeara es válido. En caso de no serlo invoca a la interrupción 0x46, que es básicamente igual a las interrupciónes que ya están configuradas (de la 0 a la 19), excepto que tiene un mensaje particular correspondiendo al error. Si el índice es válido, borra de la pantalla (en la arena) la página compartida que tenía asignada -en caso de tenerla-, mapea la nueva (mediante funciones implementadas en ejercicios anteriores) y la pinta la nueva página en la pantalla.\\

Finalmente, para articular estos servicios, implementamos en el \textit{isr.asm} y el \textit{idt.c} la interrupción \textit{\_isr69} (69d = 0x45). Su implementación chequea si en el registro \textit{eax} hay algo distinto que 286 ó 386 (ambos códigos válidos de los servicios del \textit{kernel}). En caso de haber un código de servicio inválido, se desaloja a la tarea mediante la ayuda glorificadora de \textit{chuckNorris}, o sino se invoca a la función que corresponda al servicio invocado.
	  		%\end{minipage}
		  	\pagebreak
	\section{Preguntas}
	  	\begin{minipage}{\linewidth}
	\subsection{Pregunta 1}
		\begin{minipage}{\linewidth}
			\textit{¿Qué ocurre si se intenta escribir en la fila 26, columna 1 de la matriz de video, utilizando el segmento de la GDT que direcciona a la memoria de video? ¿Por qué?}
			
			Al intentar escribir en la linea 26, columna 1, utilizando el segmento de video, de la siguiente forma:
mov [fs:26*80*2],bx
Lo que ocurre es una excepción de "General Protection". Esto ocurre porque el segemento de video tiene solo 24 lineas (24*80*2 Bytes), y al intentar escribir en una posición que excede dicho límite,  ocurre la excepción.

		\end{minipage}
\end{minipage}

\begin{minipage}{\linewidth}
	\subsection{Pregunta 2}
		\begin{minipage}{\linewidth}
			\textit{¿Qué ocurre si no se setean todos los registros de segmento al entrar en
modo protegido? ¿Es necesario setearlos todos? ¿Por qué?}

			Lo que ocurre varía según cual sea el segmento no seteado:
	fs (Video): Al intentar utilizarlo, cuando pintamos la primer y ultima línea de la pantalla de blanco, y se lanza la excepción 'INT 9: Coprocessor Segment Overrun (reserved)', por lo que no se pinta nada en la pantalla de esta forma.
	
	ss (Stack): Al llamar a la funcion inicializar\_dir\_kernel se lanza la excepción 'INT 12: Stack Segment Fault'
	
	gs,es : Podria no setearse. Se lanzan las mismas excepciones que cuando estan seteados. 
	
	ds: Se lanza la expceción 'INT 13: General Protection Interruption' al llamar a inicializar\_dir\_kernel.
	
Nota, pusimos la intruccion sti antes de pintar la pantalla para detectar que excepciones se lanzaban, pero en realidad el primer sti esta luego de resetear el pic, por lo que estas excepciones no se capturan. 
	
Concluyendo, podrian no setearse los registros gs y es. Esto no causa problema porque ninguna instrucción asume como segmento al seleccionado por alguno de estos 2 registros, a diferencia de esto, por ejemplo la instrucción 'mov' toma como predeterminado el DS.
		\end{minipage}
\end{minipage}

\begin{minipage}{\linewidth}
	\subsection{Pregunta 3}
		\begin{minipage}{\linewidth}
			\textit{¿Cómo se puede hacer para generar una excepción sin utilizar la instrucción int? Mencionar al menos 3 formas posibles.}
			
			Las excepciones son interrupciones que genera el procesador cuando algo no funciona como es debido. Por esto mismo, la respuesta a esta pregunta es, básicamente, mencionar tres problemas que pueden llegar a ocurrir a lo largo de la ejecución de un programa.

\begin{itemize}
	\item La primera manera elegida para generar una interrupción es una división por cero. Como dividir por cero no es una operación válida, el procesador lo detecta e inmediatamente arroja una excepción de tipo 0 (Divide error).
		
	\item Otra manera de producir una excepción es por ejemplo intentar escribir una página que tiene permisos únicamente de lectura (en el flag r/w de su entrada en la tabla de páginas). Esto arroja una excepción de tipo 14 (Page fault).

	\item Utilizando un selector de segmento, intentar acceder a un espacio de memoria fuera del límite de dicho límite. (como en el ejercicio 1).

\end{itemize}
		\end{minipage}
\end{minipage}

\begin{minipage}{\linewidth}
	\subsection{Pregunta 4}
		\begin{minipage}{\linewidth}
			\textit{¿Cuáles son los valores del stack cuando se genera una interrupción? ¿Qué significan? (indicar para el caso de operar en nivel 3 y nivel 0}
			
			Cuándo se arroja una excepción en nivel 3, la pila se encuentra con la información del programa al momento de terminar la ejecución. Esto es, con los registros de segmento de código y stack que se estaban utilizando, el EIP de la instrucción que se estaba ejecutando, el ESP que apunta a la pila de nivel 3, los EFLAGS y el ERRORCODE que indica que tipo de interrupción se produjo.
\begin{center}
  \begin{tabular*}{200mm}{l|c|}
  & \rule[0.25cm]{80mm}{0.01cm}\\
  esp & ERRORCODE\\
  esp + 4 & EIP(Previo a la interrupción)\\
  esp + 8 & CS\\
  esp + 12 & EFLAGS\\
  esp + 16 & ESP(Previo a la interrupción)\\
  esp + 20 & SS\\
  & \rule[-0.2cm]{80mm}{0.01cm}\\
  \end{tabular*}
\end{center}

En el caso de la excepción en nivel 0, el estado de la pila es similar, pero se evita guardar los registros que apuntan a la pila (SS y ESP) ya que al no haber cambio de nivel de privilegio, estos siguen siendo iguales.

\begin{center}
  \begin{tabular*}{200mm}{l|c|}
  & \rule[0.25cm]{80mm}{0.01cm}\\
  esp & ERRORCODE\\
  esp + 4 & EIP(Previo a la interrupción)\\
  esp + 8 & CS\\
  esp + 12 & EFLAGS\\
  & \rule[-0.2cm]{80mm}{0.01cm}\\
  \end{tabular*}
\end{center}

		\end{minipage}
\end{minipage}

\begin{minipage}{\linewidth}
	\subsection{Pregunta 5}
		\begin{minipage}{\linewidth}
			\textit{¿Puede el directorio de páginas estar en cualquier posición arbitraria de
memoria?}

			El directorio de páginas es accedido mediante el registro CR3 que tiene la dirección base del mismo. La restricción con respecto a la ubicación del directorio está relacionada con esto mismo.\\
Para acceder al directorio el procesador asume que el mismo está alineado a 4K, es decir, que está ubicado en una página y por lo tanto agrega 12 ceros a la derecha de la dirección que esté almacenada en el CR3 para obtener la dirección física final.\\

Esto implica que el directorio no puede estar almacenado en cualquier posición de memoria. Únicamente puede ubicarse en posiciones de memoria alineadas a 4K.
		\end{minipage}
\end{minipage}

\begin{minipage}{\linewidth}
	\subsection{Pregunta 6}
		\begin{minipage}{\linewidth}
			\textit{¿Es posible acceder a una página de nivel de kernel desde usuario?}
			
			Cuándo estás ejecutando código a nivel usuario, tu nivel de privilegio es 3 (usuario). Las entradas de las tablas de páginas que apuntan a una página del kernel tienen seteado el bit de \textbf{U/S} en 0.\\

Al intentar acceder a esta página, se produce una comprobación en la cuál el procesador evalúa si el nivel con el que se está ejecutando el código es superior o igual al nivel de la página a la que se desea acceder.\\
En este caso, el código se está ejecutando en nivel \textbf{U(3)} y la página requiere un nivel \textbf{S(0)} para ser accedida, lo que causa una excepción cuando el usuario intenta acceder a la misma.\\

Dicho esto, concluímos que un usuario con nivel de privilegio \textbf{U(3)} no puede acceder a ninguna página del kernel.
		\end{minipage}
\end{minipage}

\begin{minipage}{\linewidth}
	\subsection{Pregunta 7}
		\begin{minipage}{\linewidth}
			\textit{¿Se puede mapear una página física desde dos direcciones virtuales distintas, tal que una esté mapeada con nivel de usuario y la otra a nivel de kernel?, De ser posible, ¿Qué problemas puede traer?}
			
			Sí, se puede. Para lograrlo, se puede setear una entrada de la Tabla de Páginas con la misma dirección que la entrada anterior, y con el bit u/s en 1, suponiendo que la anterior tiene el bit u/s en 0.
% (Lo probé con set tbl entry((tbl entry*) PTK,(TBL COUNT/2) -1,1,1,1,0,0,0,0,0,0,(unsigned int)(TBL COUNT/2) -2);, que setea la ultima pagina de los 2 mb del kernel a la anteultima pagina fisica.)
 %
El problema que puede traer esto es que un programa puede modificar una posición de memoria que el sistema también utiliza. Dependiendo de para que la utilice el sistema, varían los problemas posibles. Por ejemplo si allí se guarda la idt, podría dejar de atenderse una interrupción.

		\end{minipage}
\end{minipage}

\begin{minipage}{\linewidth}
	\subsection{Pregunta 8}
		\begin{minipage}{\linewidth}
			\textit{¿Qué permisos pueden tener las tablas y directorios de paginas? ¿Cuáles son los permisos efectivos de una dirección de memoria según los permisos del directorio y tablas de páginas?}
			
			Ambas entradas, las de directorio de páginas y de tablas, pueden tener permisos de Sistema o de Usuario segun el bit u/s, y permisos de Lectura o Lectura-Escritura según el bit r/w.

Una dirección de memoria tiene permiso de lectura/escritura, si ambas entradas, la de directorio y la de tablas, tienen dicho bit seteado. De lo contrario, es de solo lectura. Por otro lado, puede ser utilizada por tareas que corren en nivel usuario, solamente si, nuevamente, ambas entradas tienen el bit u/s seteado. De otra forma, solo será accesible para tareas de nivel 0.

		\end{minipage}
\end{minipage}

\begin{minipage}{\linewidth}
	\subsection{Pregunta 9}
		\begin{minipage}{\linewidth}
			\textit{¿Es posible desde dos directorios de página, referenciar a una misma tabla de páginas?}
			
			Si, se puede. El resultado es que distintas direcciones virtuales se mapean a las mismas posiciones físicas.

%(Lo probe haciendo set tbl entry((tbl entry*) PTK,(TBL COUNT/2) -1,1,1,1,0,0,0,0,0,0,(unsigned int)(TBL COUNT/2) -2);)
		\end{minipage}
\end{minipage}

\begin{minipage}{\linewidth}
	\subsection{Pregunta 10}
		\begin{minipage}{\linewidth}
			\textit{¿Qué es el TLB (Translation Lookaside Buffer) y para qué sirve?}
			
			El TLB es un buffer en el que se cachea el mapeo de los 20 bits mas significativos de las direcciones lineales, a los 20 bits más significativos de las direcciones de memoria física, a las cuales se corresponden. 
Sirve para agilizar el proceso de traducción de direcciones lineales a físicas.
		\end{minipage}
\end{minipage}

\begin{minipage}{\linewidth}
	\subsection{Pregunta 11}
		\begin{minipage}{\linewidth}
			\textit{¿Qué pasa si en la interrupción de teclado no se lee la tecla presionada?}
			
			Nada. El programa sigue ejecutando normalmente, pero no se muestran por pantalla, las letras tipeadas.
%Habria que ver si hay que decir si se van almacenando en el puerto de teclado o se van reemplazando
		\end{minipage}
\end{minipage}

\begin{minipage}{\linewidth}
	\subsection{Pregunta 12}
		\begin{minipage}{\linewidth}
			\textit{¿Qué pasa si no se resetea el PIC?}
			
			Por defecto, el pic1 tiene mapeadas las interrupciones de la 0x8 a 0xF. Entonces, al no resetear (remapear) el pic, la interrupcion que entre en la primer "pata" del pic, se interpretará como la interrupción 8 y la segunda como la 9. Entonces cada vez que ocurre una interrupción de teclado o de reloj, en vez de activarse la rutina de atención de interrupciones correspondiente a ellas, se invocan a las rutinas de las interrupciones número 8 y 9 respectivamente.
		\end{minipage}
\end{minipage}

\begin{minipage}{\linewidth}
	\subsection{Pregunta 13}
		\begin{minipage}{\linewidth}
			\textit{Colocando un breakpoint luego de la cargar una tarea, ¿cómo se puede verificar, utilizando el debugger de Bochs, que la tarea se cargó correctamente? ¿Cómo se llega a esta conclusión?}
			
			Se puede verificar que la tarea se cargó, poniendo un breakpoint en la interrupción de reloj, y ejecutando 'info tss' una vez que se entró a dicha interrupción. Para confirmar que la tarea es la que nosotros queremos, se puede verificar el 'eip', 'cr3', 'tr', 'base' devueltos por la instrucción. 
Esto es valido porque al cargarse una tarea, el procesador levanta el contexto de la misma. Esto significa que pasa toda la información almacenada en la tss a los registros de contexto corresponientes, como ser los registros de propocito general, los de segmento, etc.
		\end{minipage}
\end{minipage}

\begin{minipage}{\linewidth}
	\subsection{Pregunta 14}
		\begin{minipage}{\linewidth}
			\textit{¿Cómo se puede verificar si la conmutación de tarea fue exitosa?}
			
			Se puede verificar teniendo el mismo breakpoint que en la pregunta anterior, y al momento de llegar a él, ejecutar nuevamente 'info tss', y verificar que el valor de 'tr' cambió.
		\end{minipage}
\end{minipage}

\begin{minipage}{\linewidth}
	\subsection{Pregunta 15}
		\begin{minipage}{\linewidth}
			\textit{Se sabe que las tareas llaman a la interrupción 0x45. ¿Qué ocurre si esta no está implementada? ¿Por qué?}
			
			Lo que ocurre es que al no estar seteada la entrada del vector de interrupciones que describe a esta interrupción, cuándo la misma es arrojada, el procesador intenta buscarla en un lugar de memoria inválido y por lo tanto arroja una excepción de General Protection.

		\end{minipage}
\end{minipage}
\end{document}
